\documentclass{beamer}

\mode<presentation>
{
  \usetheme{Madrid}
  % or ...

  % \setbeamercovered{transparent}
  % or whatever (possibly just delete it)
}


\usepackage[czech]{babel}
% or whatever

\usepackage[utf8]{inputenc}
% or whatever

\usepackage{times}
\usepackage[T1]{fontenc}
% Or whatever. Note that the encoding and the font should match. If T1
% does not look nice, try deleting the line with the fontenc.

\usepackage{tikz,tikz-qtree}

\title{Persistentní weak-AVL stromy}
\subtitle{Persistent Weak-AVL Trees}

\author
{Jiří Škrobánek}
% - Give the names in the same order as the appear in the paper.
% - Use the \inst{?} command only if the authors have different
%   affiliation.


\date{2. červenec 2021}

% If you have a file called "university-logo-filename.xxx", where xxx
% is a graphic format that can be processed by latex or pdflatex,
% resp., then you can add a logo as follows:

\pgfdeclareimage[height=0.5cm]{university-logo}{../img/logo-cz}
\logo{\pgfuseimage{university-logo}}


% If you wish to uncover everything in a step-wise fashion, uncomment
% the following command: 

%\beamerdefaultoverlayspecification{<+->}



\newtheoremstyle{mytheoremstyle} % name
{\topsep}                    % Space above
{\topsep}                    % Space below
{\slshape}                   % Body font
{}                           % Indent amount
{\bfseries}                   % Theorem head font
{.}                          % Punctuation after theorem head
{.5em}                       % Space after theorem head
{}  % Theorem head spec (can be left empty, meaning ‘normal’)

\theoremstyle{mytheoremstyle}
\newtheorem{dfn}{Definice}
\newtheorem{thm}{Věta}

\usetikzlibrary{shapes.geometric}
\usetikzlibrary{positioning}
\tikzset{every tree node/.style={minimum size=7mm,draw,circle},
         blank/.style={draw=none},
         edge from parent path={(\tikzparentnode)--(\tikzchildnode)},
         level distance=1.2cm,
         triangle/.style={regular polygon, regular polygon sides=3},
         tosubtree/.style={edge from parent path={(\tikzparentnode)--(\tikzchildnode.north)}},
         }


\begin{document}

\begin{frame}
  \titlepage
\end{frame}

\begin{frame}{Zadání}
\begin{itemize}
\item
Driscoll et al.~upravili červeno-černé stromy a získali tak persistentní vyhledávací stromy.
\item
%Weak-AVL trees possess interesting properties for persistence.
Weak-AVL stromy mají některé vlastnosti indikující, že by bylo možné provést podobou úpravu.
\item
Úkolem je prozkoumat, zda je možné {\bf postavit efektivní persistentní weak-AVL strom} se složitostí amortizovaně logaritmickou na operaci.
%Investigate if weak-AVL trees could be modified similarly to red-black trees and used in persistence.
\end{itemize}
\end{frame}


\begin{frame}{Hierarchie persistence}
\begin{itemize}
  \item \textbf{Efemerní} jsou běžné datové struktury.
  \item \textbf{Semi-persistentní} datové struktury navíc zachovávají pro čtení historii. Lineární uspořádání verzí.
  \item \textbf{(Plně) persistentní struktury} navíc připouštějí změny na starších verzí. Historie je strom.
  \item \textbf{Funkcionální struktury} navíc nepřipouštějí žádné změny existujících verzí struktury. 
\end{itemize}
Cílíme na plně persistentní strukturu.
\end{frame}



\begin{frame}{Rankově vyvážené stromy}
\begin{dfn}[Rankově vyvážený strom]
Binární strom je rankově vyvážený, pokud je každému (vnitřnímu) vrcholu přiděleno přirozené číslo jako \emph{rank}. Externí vrcholy mají rank $-1$.
%Binary tree is said to be ranked-balanced if all vertices have non-negative integer ranks assigned. Externals vertices are assigned -1 by convention.
\end{dfn}
Koncept představili Haeupler et al.

\pause

Typicky vynucujeme konkrétní \emph{hranová pravidla}. Pokud má vrchol o $a$ vyšší rank než jeden syn a o $b$ než druhý, označíme ho jako $(a,b)$ vrchol.

\pause

\begin{thm}[Hloubka rankově vyvážených stromů]
Pokud je v rankově vyváženém stromě $n$ vrcholů a všechny vrcholu mají typ z nějaké konečné množiny $R \subset \mathbb{N}^2$, hloubka stromy je $\Theta(\log n)$.
\end{thm}
\end{frame}

\begin{frame}{Weak-AVL stromy}
\begin{dfn}[Weak-AVL strom]
Rankově vyvážený strom je weak-AVL, pokud všechny jeho vrcholy jsou $(1,1)$, $(2,1)$, nebo $(2,2)$.
\end{dfn}

\pause

Každý AVL strom je možné orankovat, aby měly vrcholy ranky $(1,1)$ nebo $(2,1)$.

\pause

Existují dvě varianty vyvažování (shora, zespoda). 

Vyvažování shora se děje preventivě a má horší konstanty. Dá se ale využít pro paralelní semi-persistentní strukturu. (Vedlější výsledek.)
\end{frame}

\begin{frame}
\frametitle{Vyvažování zespoda po vkládání}

\only{Vyvažování probíhá směrem ke koření, dokud je porušen rankový invariant.}<1>

\only{Sekvence povyšování ranků potenciálně končící (dvojitou) rotací.}<1>

\begin{center}

\only{
\begin{tikzpicture}[sibling distance=32pt]
  \Tree
  [ \edge node[auto=right]{1 or 2}; [.$y$
      \edge[tosubtree] node[auto=right] {0};
      \node[draw,triangle]{~};
      \edge[tosubtree] node[auto=left] {1};
      \node[draw,triangle]{~};
  ] ]
  \end{tikzpicture}
  \qquad\hspace{20mm}
  \begin{tikzpicture}[sibling distance=32pt]
  \Tree
  [ \edge node[auto=right]{0 or 1}; [.$y$
      \edge[tosubtree] node[auto=right] {1};
      \node[draw,triangle]{~};
      \edge[tosubtree] node[auto=left] {2};
      \node[draw,triangle]{~};
  ] ]
\end{tikzpicture}
}<1>

\only{
  \begin{tikzpicture}[sibling distance=8pt, 
    frontier/.style={distance from root=4cm}]
  \Tree
  [ \edge node[auto=right]{1 or 2}; [.$y$
      \edge[very thick] node[auto=right] {0};
      [   .$x$ 
          \edge[tosubtree] node[auto=right] {1}; 
          \node[draw,triangle]{A}; 
          \edge[tosubtree] node[auto=left] {2};
          \node[draw,triangle]{B}; ]
      \edge[tosubtree] node[auto=left] {2};
      \node[draw,triangle]{C};
  ] ]
  \end{tikzpicture}
  \qquad
  \begin{tikzpicture}[sibling distance=8pt, 
    frontier/.style={distance from root=4cm}]
  \Tree
  [ \edge node[auto=right]{1 or 2}; [.$x$
      \edge[tosubtree] node[auto=right] {1};
      \node[draw,triangle]{A};
      \edge[very thick] node[auto=left] {1};
      [   .$y$ 
          \edge[tosubtree] node[auto=right] {1}; 
          \node[draw,triangle]{B}; 
          \edge[tosubtree] node[auto=left] {1}; 
          \node[draw,triangle]{C}; ]
  ] ]
  \end{tikzpicture}
}<2>

\only{

\small

\begin{tikzpicture}[sibling distance=4pt, frontier/.style={distance from root=5.25cm}]
\Tree
[ 
    \edge node[auto=right]{1 or 2}; 
    [
        .$y$
        \edge[very thick] node[auto=right] {0};
        [   
            .$x$ 
            \edge[tosubtree] node[auto=right] {2}; 
            \node[draw,triangle]{A}; 
            \edge[very thick] node[auto=left] {1};
            [
                .$w$
                \edge[tosubtree]; 
                \node[draw,triangle]{B};
                \edge[tosubtree]; 
                \node[draw,triangle]{C}; 
            ]
        ]
        \edge[tosubtree] node[auto=left] {2};
        \node[draw,triangle]{D};
    ] 
]
\end{tikzpicture}
\qquad
\begin{tikzpicture}[sibling distance=4pt, frontier/.style={distance from root=4cm}]
\Tree
[ \edge node[auto=right]{1 or 2}; [.$w$
    \edge[very thick] node[auto=right] {1};
    [ 
        .$x$
        \edge[tosubtree] node[auto=right] {1};
        \node[draw,triangle]{A};
        \edge[tosubtree];
        \node[draw,triangle]{B};
    ]
    \edge[very thick] node[auto=left] {1};
    [   .$y$ 
    	\edge[tosubtree];
        \node[draw,triangle]{C}; 
        \edge[tosubtree] node[auto=left] {1}; 
        \node[draw,triangle]{D}; ]
] ]
\end{tikzpicture}
}<3>

\end{center}

\end{frame}


\begin{frame}
\frametitle{Vyvažování zespoda po odstranění}
  
\only{Sekvence ponižování ranků potenciálně končící (dvojitou) rotací.}<1>

\begin{center}

\only{
\begin{tikzpicture}[sibling distance=32pt, frontier/.style={distance from root=2cm}]
  \Tree
  [ \edge node[auto=right]{1 or 2}; [.$y$
  \edge[tosubtree] node[auto=right] {3};
  \node[draw,triangle]{~};
  \edge[tosubtree] node[auto=left] {2};
  \node[draw,triangle]{~};
  ] ]
  \end{tikzpicture}
  \qquad
  \begin{tikzpicture}[sibling distance=32pt, frontier/.style={distance from root=2cm}]
  \Tree
  [ \edge node[auto=right]{2 or 3}; [.$y$
  \edge[tosubtree] node[auto=right] {2};
  \node[draw,triangle]{~};
  \edge[tosubtree] node[auto=left] {1};
  \node[draw,triangle]{~};
  ] ]
\end{tikzpicture}

\vspace{10mm}

\small

\begin{tikzpicture}[sibling distance=8pt, frontier/.style={distance from root=3.5cm}]
  \Tree[ 
  \edge node[auto=right] {1 or 2};
  [   
  .$x$ 
  \edge[tosubtree] node[auto=right] {3}; 
  \node[draw,triangle]{A}; 
  \edge node[auto=left] {1};
  [
  .$y$
  \edge[tosubtree] node[auto=right] {2};
  \node[draw,triangle]{B};
  \edge[tosubtree] node[auto=left] {2};
  \node[draw,triangle]{C}; 
  ] ] ]
  \end{tikzpicture}
  \qquad
  \begin{tikzpicture}[sibling distance=8pt, frontier/.style={distance from root=3.5cm}]
  \Tree[ 
  \edge node[auto=right] {2 or 3};
  [   
  .$x$ 
  \edge[tosubtree] node[auto=right] {2}; 
  \node[draw,triangle]{A}; 
  \edge node[auto=left] {1};
  [
  .$y$
  \edge[tosubtree] node[auto=right] {1};
  \node[draw,triangle]{B};
  \edge[tosubtree] node[auto=left] {1};
  \node[draw,triangle]{C}; 
  ] ] ]
  \end{tikzpicture}
}<1>


\only{
  \begin{tikzpicture}[sibling distance=16pt, frontier/.style={distance from root=4cm}]
    \Tree
    [ \edge node[auto=right]{1 or 2}; [.$y$
        \edge[tosubtree] node[auto=right] {3};
        \node[draw,triangle]{A};
        \edge[very thick] node[auto=left] {1};
        [   .$x$ 
            \edge[tosubtree] node[auto=right] {1 or 2}; 
            \node[draw,triangle]{B}; 
            \edge[tosubtree] node[auto=left] {1};
            \node[draw,triangle]{C}; ]
    ] ]
    \end{tikzpicture}
    \qquad
    \begin{tikzpicture}[sibling distance=16pt, frontier/.style={distance from root=4cm}]
    \Tree
    [ \edge node[auto=right]{1 or 2}; [.$x$
        \edge[very thick] node[auto=right] {1};
        [ .$y$ 
            \edge[tosubtree] node[auto=right] {2}; 
            \node[draw,triangle]{A}; 
            \edge[tosubtree] node[auto=left] {1 or 2}; 
            \node[draw,triangle]{B}; ]
        \edge[tosubtree] node[auto=left] {2};
        \node[draw,triangle]{C};
    ] ]
    \end{tikzpicture}
}<2>

\small

\only{
\begin{tikzpicture}[sibling distance=4pt, frontier/.style={distance from root=5.25cm}]
\Tree
[ 
    \edge node[auto=right]{1 or 2}; 
    [
        .$y$
        \edge[tosubtree] node[auto=right] {3};
        \node[draw,triangle]{A};
        \edge[very thick] node[auto=left] {1};
        [   
            .$x$ 
            \edge[very thick] node[auto=right] {1};
            [
                .$w$
                \edge[tosubtree];
                \node[draw,triangle]{B};
                \edge[tosubtree];
                \node[draw,triangle]{C}; 
            ]
            \edge[tosubtree] node[auto=left] {2}; 
            \node[draw,triangle]{D}; 
        ]
    ] 
]
\end{tikzpicture}
\qquad
\begin{tikzpicture}[sibling distance=4pt, frontier/.style={distance from root=4cm}]
\Tree
[ \edge node[auto=right]{1 or 2}; [.$w$
    \edge[very thick] node[auto=right] {2};
    [ 
        .$y$
        \edge[tosubtree] node[auto=right] {1};
        \node[draw,triangle]{A};
        \edge[tosubtree];
        \node[draw,triangle]{B};
    ]
    \edge[very thick] node[auto=left] {2};
    [   .$x$ 
    	\edge[tosubtree];
        \node[draw,triangle]{C}; 
        \edge[tosubtree] node[auto=left] {1}; 
        \node[draw,triangle]{D}; ]
] ]
\end{tikzpicture}
}<3>

\end{center}

\end{frame}

\begin{frame}{Konstrukce persistence}
Chceme použít obecné schéma pro konstrukci persistentních pointerových struktur. (Driscoll et al.)

\begin{itemize}
\item
Souvislý interval verzí je reprezentován \emph{tlustým vrcholem}. 
\item 
Tlustý vrchol obsahuje $\Theta(1)$ informace o sousedech (i přidané pointery) a hodnoty vrcholu.
\item
Když přeteče kapacita tlustého vrcholu, proběhne dělící fáze.
\end{itemize}

\pause

\begin{block}{Předpoklady}
\begin{itemize}
  \item {\color{green!80!black}Omezené stupně vrcholů.}
  \item {\color{red}Omezený počet změn struktury v jedné operaci.}
  \item Uspořádání verzí. (Váhově vyváženým stromem)
\end{itemize}
\end{block}

\end{frame}

\begin{frame}{Spojování změn}
\begin{itemize}
\item Zavedeme cestové změny pro sérii povýšení nebo ponížení.
\item Cesta bude označena pouze horním vrcholem.
\item Cesty se mohou rozdělit na dvě části.
\item Vrchol bude ležet nejvýše na dvou cestách najednou.
\item Spousta detailů...
\end{itemize}

\pause

\begin{thm}
Pozměněný algoritmus vyvažující weak-AVL stromy potřebuje $\Theta(1)$ změn na operaci.
\end{thm}

Nový Weak-AVL strom můžeme použít pro persistenci.
\end{frame}

\begin{frame}

\centering

\only{
  \begin{tikzpicture}[sibling distance=8pt]
    \Tree
    [
    .3
    [ 
        .2
        [ .1 0 0 ]
        [ .1 0 0 ]
    ] 
    [ 
        .2
        [ .1 0 0 ]
        [ .1 0 0 ]
    ] 
    ]
    \end{tikzpicture}
  }<1>

  \only{
  \begin{tikzpicture}[sibling distance=8pt]
    \Tree
    [
    .3
    [ 
        .2
        [ .1 0 0 ]
        [ .1  [.\node[red]{0}; 0 \edge[blank]; \node[blank]{}; ] 0 ]
    ] 
    [ 
        .2
        [ .1 0 0 ]
        [ .1 0 0 ]
    ] 
    ]
    \end{tikzpicture}
  }<2>

  \only{
    \begin{tikzpicture}[sibling distance=8pt]
      \Tree
      [
      .4
      [ 
          .3
          [ .1 0 0 ]
          [ .2  [.1 0 \edge[blank]; \node[blank]{}; ] 0 ]
      ] 
      [ 
          .2
          [ .1 0 0 ]
          [ .1 0 0 ]
      ] 
      ]
      \end{tikzpicture}
    }<3>

  \only{
    \begin{tikzpicture}[sibling distance=8pt]
    \Tree
    [
    .\node[very thick]{3};
    \edge[very thick] node[auto=right] {~};
    [ 
        .\node[very thick]{2};
        [ .1 0 0 ]
        \edge[very thick] node[auto=right] {~};
        [  .\node[very thick]{1}; \edge[very thick] node[auto=right] {~}; [.\node[very thick]{0}; 0 \edge[blank]; \node[blank]{}; ] 0 ]
    ] 
    [ 
        .2
        [ .1 0 0 ]
        [ .1 0 0 ]
    ]
    ]
  \end{tikzpicture}
  }<4>

  \only{
    \begin{tikzpicture}[sibling distance=8pt]
    \Tree
    [
    .\node[very thick]{3};
    \edge[very thick] node[auto=right] {~};
    [ 
        .\node[very thick]{2};
        [ .1 0 0 ]
        \edge[very thick] node[auto=right] {~};
        [  .\node[very thick]{1}; \edge[very thick] node[auto=right] {~}; \node[very thick,red]{0}; 0 ]
    ] 
    [ 
        .2
        [ .1 0 0 ]
        [ .1 0 0 ]
    ]
    ]
  \end{tikzpicture}
  }<5>


  \only{
    \begin{tikzpicture}[sibling distance=8pt]
    \Tree
    [
    .\node[very thick]{3};
    \edge[very thick] node[auto=right] {~};
    [ 
        .\node[very thick]{2};
        [ .1 0 0 ]
        \edge[very thick] node[auto=right] {~};
        [  .\node[very thick]{1}; 0 0 ]
    ] 
    [ 
        .2
        [ .1 0 0 ]
        [ .1 0 0 ]
    ]
    ]
  \end{tikzpicture}
  }<6>
\end{frame}

\begin{frame}{Výsledky}
\begin{enumerate}
\item
Reprezentace weak-AVL stromu s konstantním počtem změn na operaci v nejhorším případě
\item
Persistentní Weak-AVL strom (Bohužel se zdá být složitější než persistentní červeno-černé stromy.)
\item
Proof of concept implementace
\item
Modifikovaná konstrukce persistentních pointerových struktur
\item
Semi-persistentní weak-AVL strom s paralelnímy operacemi
\end{enumerate}
\end{frame}

\end{document}
