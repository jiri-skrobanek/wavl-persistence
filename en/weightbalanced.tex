\section{Weight-Balanced Trees}

In our algorithm for persistent binary tree, we will also need weight-balanced (originally BB[$\alpha$]) trees. This kind of data structure was invented by J. Nievergelt and E. M. Reingold \cite{weight-balanced}. Here we describe properties of a slightly improved version of weight-balanced trees. The original structure used rotations for balancing and would not be usable in this context.

Weight-balanced trees form a class of binary search trees. The strategy of maintaining balance in weigh-balanced trees uses a different approach from that of rank-balanced trees. The idea stands as follows: When \emph{weight} is the same for left and right subtree, then it decreases exponentially along vertical paths, guaranteeing logarithmic depth.

A vertex of a weight-balanced tree has the following entries stored (in addition to key and value): {\em size} of subtree rooted at that vertex and pointers to {\em left} and {\em right} child. By definition the size of external vertex is 0. {\em Weight} of a vertex is then defined as size + 1.

\begin{defn}
Let $\alpha \in (0,1/2)$ be constant. We call a binary search $T$ an {\em $\alpha$-weight-balanced tree} if it holds for every vertex $v$ of T that $\weightit(left(v)) \geq \alpha \cdot \weightit(v) \wedge \weightit(\rightit(v)) \geq \alpha \cdot \weightit(v) $.
\end{defn}

\begin{prop}
Let $\alpha \in (0,1/2)$ be constant. Every $\alpha$-weight-balanced tree $T$ with $n$ vertices has depth $\Theta(\log n)$. 
\end{prop}

\begin{myproof}
Let us consider an inner vertex of size $s$ with children of sizes $s_l$ and $s_r$. Then from the definition of weight-balanced trees:

$$s_l + 1 \geq \alpha (s_r+1)$$
$$ s = s_l + s_r + 1 $$
$$ s \geq \alpha (s_r+1) + s_r $$
$$ s \geq (1+\alpha)s_r $$

Which implies that size of parent is at least a constant factor greater than of each of its children.
\end{myproof}

Now we turn to the algorithm for insertion. (Deletion is similar and will not be discussed as we do not need it in this thesis). As usual we determine the place of insertion and insert the vertex. Next we update sizes of all vertices on the path toward the root. This can break balance of some vertices on this path, with $v$ being closest to the root. If $v$ exists, the entire subtree rooted at $v$ is reconstructed into a perfectly balanced binary tree.

It can be proven that the asymptotic cost of this insert is logarithmic.

\begin{prop}
Any $n$ consecutive operations performed on initially empty weight-balanced tree have total complexity \bigO{n \log n}. 
\end{prop}

\begin{myproof}
We define potential of each vertex $v$ to be $c \cdot |\sizeit(\leftit(v)) - \sizeit(\rightit(v))|$ for suitable constant $c$ dependent on $\alpha$ and computational model. (Naturally external vertices have size equal to 0.) Sum of potentials over the whole tree is initially 0 and may never become negative.

One insert may increase potential of all vertices on the path connecting it to root vertex. This is bounded by a constant multiple of size of the tree by previous proposition.

The initial insertion phase before rebalancing takes \bigO{n \log n} time. We will also charge the insert with increase in potential caused by it. Rebalancing at vertex $x$ will be paid from the decrease of potential of $x$ by rebalancing. 

It remains to show that the saved potential at $v$ is sufficient to pay for rebalancing at $v$. This is easily seen. 

Without loss of generality
$$ \sizeit(\leftit(v)) + 1 < \alpha ( \sizeit(\leftit(v)) + \sizeit(\rightit(v)) + 1 ) $$
holds before rebalancing, which yields
$$ (\sizeit(\leftit(v)) + 1)(1 - \alpha) < \alpha \sizeit(\rightit(v)), $$
giving
$$ { \sizeit(\leftit(v)) + 1 \over \sizeit(\rightit(v)) } < { \alpha \over 1 - \alpha }. $$

From the fact that weight invariant did not hold for $v$ we can deduce, that either the subtree has size limited by a constant or its potential must be on the order of $\Theta(\sizeit(v))$, which is sufficient to pay for the reconstruction of this subtree in \bigO{size(v)}. 
\end{myproof}