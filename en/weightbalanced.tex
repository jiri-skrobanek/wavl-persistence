\section{Weight-Balanced Trees}

In our algorithm for persistent binary tree, we will also need weight-balanced (originally BB[$\alpha$]) trees. This kind of data structure was invented by J. Nievergelt and E. M. Reingold \cite{weight-balanced}. Here we describe properties of a slightly improved version of weight-balanced trees. The original structure used rotations for balancing and would not be usable in this context.

Weight-balanced trees form a class of binary search trees. The strategy of maintaining balance in weigh-balanced trees uses a different approach from that of rank-balanced trees. The idea stands as follows: When \emph{weight} is the same for left and right subtree, then it decreases exponentially along vertical paths, guaranteeing logarithmic depth.

A vertex of a weight-balanced tree contains the {\em size} of the subtree rooted at that vertex and pointers to. That is of course in addition to key, value and pointers to {\em left} and {\em right} child. By definition the size of external vertex is 0. {\em Weight} of a vertex is then defined as size plus one.

\begin{defn}
Let $\alpha \in (0,1/2)$ be constant. We call a binary search $T$ an {\em $\alpha$-weight-balanced tree (BB[$\alpha$] tree)} if it holds for every vertex $v$ of T that $\weightit(\leftit(v)) \geq \alpha \cdot \weightit(v) \wedge \weightit(\rightit(v)) \geq \alpha \cdot \weightit(v) $.
\end{defn}

\begin{prop}
Let $\alpha \in (0,1/2)$ be a constant. Every $\alpha$-weight-balanced tree $T$ with $n$ vertices has depth $\Theta(\log n)$. 
\end{prop}

\begin{myproof}
Let us consider an inner vertex $v$ of size, then directly from the definition of weight-balanced trees:

$$ \sizeit(\leftit(v)) + 1 \geq \alpha (\sizeit(\rightit(v))+1)$$
$$ \sizeit(v) = \sizeit(\leftit(v)) + \sizeit(\rightit(v)) + 1 $$

Now, substituting for size of left child of $v$ into the second relation we get

$$ \sizeit(v) \geq \alpha (\sizeit(\rightit(v)) + 1) + \sizeit(\rightit(v)) $$

and

$$ \sizeit(v) \geq (1+\alpha)\sizeit(\rightit(v)).$$

This implies that size of parent is at least a constant factor greater than of each of its children.
\end{myproof}

Now we turn our attention to the algorithm for insertion. (Deletion is similar and will not be discussed as we do not need it in this thesis). As usually we determine the place of insertion and insert the vertex. Next we update sizes of all vertices on the path toward the root. This can break balance of some vertices on this path, with $v$ being closest to the root. If $v$ exists, the entire subtree rooted at $v$ is reconstructed into a perfectly balanced binary tree (via recursive division into approximately equal parts).

It can be proven that the asymptotic cost of this insert is logarithmic.

\begin{prop}
Any $n$ consecutive operations performed on an initially empty weight-balanced tree have total complexity \bigO{n \log n}. 
\end{prop}

\begin{myproof}
We define potential of each vertex $v$ to be $$c \cdot \max\left(|\sizeit(\leftit(v)) - \sizeit(\rightit(v))| - 1, 0\right)$$ for a suitable constant $c$ dependent on $\alpha$ and the computational model. (Naturally external vertices have size equal to 0.) The sum of potentials over the whole tree is initially 0 and may never become negative.

One insert may increase potential of all vertices on the path connecting it to root vertex, by up to $c$ at each vertex. Therefore the increase is bounded by a constant multiple of height of the tree.

The initial insertion phases before rebalancing combined take \bigO{n \log n} time. We will also charge the insert with increase in potential caused by it. Rebalancing at vertex $x$ will be paid from the decrease in potential of $x$ by rebalancing. 

It remains to show that the saved potential at $v$ is sufficient to pay for rebalancing at $v$. This is easily seen. 

Without loss of generality
$$ \sizeit(\leftit(v)) + 1 < \alpha ( \sizeit(\leftit(v)) + \sizeit(\rightit(v)) + 1 ) $$
holds before rebalancing, which yields
$$ (\sizeit(\leftit(v)) + 1)(1 - \alpha) < \alpha \sizeit(\rightit(v)), $$
giving
$$ { \sizeit(\leftit(v)) + 1 \over \sizeit(\rightit(v)) } < { \alpha \over 1 - \alpha }. $$

From the fact that the weight invariant did not hold for $v$ we can deduce, that either the subtree has size limited by a constant or its potential must be on the order of $\Theta(\sizeit(v))$, which is sufficient to pay for the reconstruction of this subtree in \bigO{size(v)}. After reconstruction, potential of all vertices in the subtree is zero.

We can conclude that it is sufficient to deposit \bigO{n \log n} into the potential across all operations, thus obtaining the proposition.
\end{myproof}