\begin{figure}
    \begin{center}
    \begin{tikzpicture}[sibling distance=8pt]
    \Tree
    [
    .~
    \edge node[auto=right] {1};
    [ 
        .~
        \edge node[auto=right] {1};
        [  .~ \edge node[auto=right] {1}; [.~ \edge node[auto=right] {0}; $x$ \edge[blank]; \node[blank]{}; ] \edge node[auto=left] {1}; ~ ]
        \edge node[auto=left] {1};
        [ .~ \edge node[auto=right] {1}; ~ \edge node[auto=left] {1}; ~ ]
    ] 
    \edge node[auto=left] {1};
    [ 
        .~
        \edge node[auto=right] {1};
        [  .~ \edge node[auto=right] {1}; ~ \edge node[auto=left] {1}; ~ ]
        \edge node[auto=left] {1};
        [ .~ \edge node[auto=right] {1}; ~ \edge node[auto=left] {1}; ~ ]
    ]
    ]
    \end{tikzpicture}
    \qquad
    \begin{tikzpicture}[sibling distance=8pt]
    \Tree
    [
    .\node[very thick]{~};
    \edge[very thick] node[auto=right] {1};
    [ 
        .\node[very thick]{~};
        \edge[very thick] node[auto=right] {1};
        [  .\node[very thick]{~}; \edge[very thick] node[auto=right] {1}; [.\node[very thick]{~}; \edge node[auto=right] {1}; $x$ \edge[blank]; \node[blank]{}; ] \edge node[auto=left] {2}; ~ ]
        \edge node[auto=left] {2};
        [ .~ \edge node[auto=right] {1}; ~ \edge node[auto=left] {1}; ~ ]
    ] 
    \edge node[auto=left] {2};
    [ 
        .~
        \edge node[auto=right] {1};
        [  .~ \edge node[auto=right] {1}; ~ \edge node[auto=left] {1}; ~ ]
        \edge node[auto=left] {1};
        [ .~ \edge node[auto=right] {1}; ~ \edge node[auto=left] {1}; ~ ]
    ]
    ]
    \end{tikzpicture}
    \end{center}
    \caption{An example of a promotion path creation}
    The vertex $x$ has just been inserted into the upper tree. 
    A promotion path (thick) has been created during rebalancing in the lower tree. 
    We see that this pattern can be extended to make the promotion path arbitrarily long.
    \label{fig:promotion-path}
    \end{figure}
