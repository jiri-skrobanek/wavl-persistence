For the purposes of this thesis, binary search tree will be a data structure composed of internal vertices (or nodes). Every internal vertex carries a key. Keys are members of a set of linearly ordered items and naturally extend this ordering to vertices. In addition, internal vertices are allowed to carry constant amount of information in fields (for example about tree structure) and to have constant number of pointers to other vertices. An internal node has two pointers to child vertices -- left and right. Any of the two can be an external vertex. External vertices are not represented in memory, all pointers to external vertices are equal and must not be dereferenced. For the specific application, it can be beneficial to assume external vertices have some default values of their fields. An internal left child must precede its parent in the linear order, conversely an internal right child must come after its parent in the linear order.

In this chapter, we will discuss some variants of binary search trees useful for persistence.