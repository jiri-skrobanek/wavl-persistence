\section{Alternative WAVL Tree Balancing Algorithm}

For the purposes of full persistence, standard algorithm for working with WAVL trees cannot be used. It is guaranteed that no more than one rotation occurs during insert or delete. Nonetheless, the number of rank changes can be up to $O(\log n)$. 

Fortunately, the algorithm can be modified in such a way that no operation requires more than $O(1)$ changes. It will come with the price of substantially more complicated rules for balancing though. Given that WAVL trees compared to AVL trees already have noticeably more complicated implementation, this makes the structure ill suited for the use as basic binary search tree. 

The problems in WAVL trees arose from promote and demote operations which can propagate along a path from the inserted (respectively deleted) vertex all the way to the root. Fortunately, there is a remedy for this. 

\subsection{Modifying paths}

Let us refer to a path where the for every vertex the following vertex is its child as a {\em vertical path}.
We will call the vertical path of vertices that had their ranks increased by promote during one insertion a {\em promotion path}. 
Similarly, vertical path of vertices demoted during one deletion will be referred to as a {\em demotion path}.
Let us refer to them both as {\em modifying paths}.
Of course some vertices will have their ranks changed without being on a modifying path (as a consequence of a rotation).

We will also be able to recognize where the second kind of demote took place, so the children of vertices on newly created demotion path, which were also demoted by the second kind of demote step, will not be modified. The rank change will rather be inferred from the demotion path. 

Our modified algorithm will try to handles as much changes with the help of modifying paths. Existence of a modifying path can by indicated merely at the top vertex of such a path. If we can group all but a constant number of changes during an operation into a modifying path, we can limit the number changes to the tree to a constant. With the algorithm described below, only a constant number of such paths is created or modified using the standard balancing rules during an insert or delete operation. With the required rotations and change in ranks of a constant number of vertices, these are the only changes performed by our alternative algorithm in one operation.

Let us not worry about representation for a moment, while we introduce rules for working with modifying paths.

\subsection{Modifying path creation and modification rules}

When a vertex is a part of modifying path, its children's ranks must not change (or even their children in case of demotion that also changes rank of one children). 

\begin{enumerate}

\item If there is a sequence of at least two promote steps without a rotation at the end, those promoted vertices are added to a new promotion path. 

\item If there is a sequence of at least three promote steps with a rotation at the end, the promoted vertices except the one on top are added to a new promotion path.

\item If there is a sequence of at least two demote steps without a rotation at the end, the demoted vertices (not counting children demoted by the second kind of demotion) are added to a new demotion path.

\item If there is a sequence of at least three demote steps with a rotation at the end, the demoted vertices (not counting children demoted by the second kind of demotion) except the one on top are added to a new demotion path.

\item If a vertex must be removed from a path, it is achieved by splitting the paths the vertex lies on in two parts (top and bottom), the vertex itself is part of neither. If a part obtained by splitting would have a single vertex, it is not created.

\item If a vertex would be rotated, it must be removed from all paths it is on.

\item If a vertex changes rank relative to its parent, but rank invariant at parent is kept, the parent must be removed from all paths it is on.

\item If a child had rank decreased by demote of second kind on parent and rank of one of its children changes, parent must be removed from the path it was put on by that demote (if it is still on that path).

\item No other path operations than those described here can be performed.

\end{enumerate}

We proceed by showing some properties of these rules.

\begin{obs}
Every modifying path must be at least two vertices long.
\end{obs}

\begin{obs}
When a vertex had its rank decreased by a demote of second kind on its parent, until parent is removed from its path, it will be a (1,1) vertex and cannot enter a new modifying path. 
\end{obs}

\begin{prop}
At any given time between operations, any vertex is either a part of one promotion path, one or two demotion paths, or no modifying path at all.
\label{thm-constant-paths}
\end{prop}

\begin{myproof}
We prove the claim by induction on the number of operations. At the beginning the proposition definitely holds for a tree with one vertex. 
Let us ignore symmetric cases for this analysis.\\
For insert operation: We follow the balancing process. A new leaf $l$ is added, this may break the rank invariant for its parent. Suppose $l$ or any vertex that had its rank increased and consider its parent $p$, now $p$ must be either (1,2), (1,1), (0,2) or (0,1) vertex. 
If it is a (1,2) vertex, no additional changes are needed or possible. $p$ cannot be on a path and its parent was not demoted recently by second kind demotion.\\
If it is a (1,1) vertex, it could be a member of a modifying path and would be removed from it. $p$'s parent was not demoted recently by second kind demotion.\\
If it is a (0,2) vertex, a rotation follows and all vertices taking part in it must be removed from all paths they may be on. $p$'s parent was not demoted recently by second kind demotion.\\
If it is a (0,1) vertex, it cannot be on any path. Possibly $p$'s parent could be on a demotion path due to a demote of second kind, in this case a rotation will directly follow at $p$'s parent. We add $p$ to list of candidates for a new promotion path from the first or second rule. Next $p$'s parent may have invalid invariant and we must consider options there, they are the same as for $p$. Ultimately we reach root of the tree a vertex where the invariant holds.\\
If there is a sufficient amount candidate of vertices for a path from the first of second rule, the path is created. We know that the candidate vertices were not on any path so adding them to a promotion path does not violate the proposition.\\
For delete operation: Again we follow the balancing process. A leaf is deleted, the situation at parent $p$ is one of the following.\\
If it is a (3,2) vertex, rank of $p$ is decreased by one. $p$ cannot be on any path. We add $p$ to list of candidates for a new demotion path from the third or fourth rule. Next $p$'s parent may have invalid invariant and we must consider options there, they are the same as for $p$. Ultimately we reach root of the tree a vertex where the invariant holds.
If it is a (3,1) vertex, rotation may be necessary, in that case all vertices taking part in it are removed from all paths. Otherwise demote of the second time is needed, we again inspect changes at $p$'s parent. $p$ becomes a candidate for a demotion path. If demote is needed here, $p$ may lie on a demotion path (but not a promotion path). It cannot lie on two demotion paths as it would be a (3,1)-vertex and its 1-son would be a (1,1)-vertex (From frozen ranks of its children).
If it is a (2,2) vertex, it could be a member of a modifying path and would be removed from it. $p$'s parent was not demoted recently by second kind demotion.\\a rotation follows and all vertices taking part in it must be removed from all paths they may be on. $p$'s parent was not demoted recently by second kind demotion.\\
If it is a (2,1) vertex, it cannot be on any path. Possibly $p$'s parent could be on a demotion path due to a demote of second kind, if so, it must be removed from that path.\\
If there is a sufficient amount candidate of vertices for a path from the first of second rule, the path is created. We know that the candidate vertices were not on a promotion path or two demotion paths so adding them to a promotion path does not violate the proposition.\\
\end{myproof}

\begin{prop}
Only a constant number of modifying path changes ever arises during one tree operation (insert/delete). Path creation, splitting, and deletion are counted as changes.
\end{prop}

\begin{myproof}
The statement follows from the properties of prescribed rules above. To modify a path, some of its vertices must change. The state of vertices in only at most constant distance from the walking path is modified. Up to one new modifying path is created. Other paths can be split or deleted only if they have a vertex in constant distance from the bottom of walking path or top of the new modifying path. Proposition \ref{thm-constant-paths} limits the number of such other paths to a constant.
\end{myproof}

\begin{prop}
For any two demotion paths in a tree that share vertices, one must be a subset of vertices of the other.
\end{prop}

\begin{myproof}
A demotion path cannot have its bottom vertex in the middle of another path created earlier. It would mean that during a balancing step, a vertex with a 3-child was part of a path. Inner vertices on a demotion path are (2,1)-vertices with the 2-child being the vertex on the same demotion path. This would imply a change of rank by two, which is not possible. After a change by one, the demotion path would be split.\\
A demotion path's upward expansion during balancing stops where rotation is needed or when rank invariant holds. Rotation ensures removal from all paths, so the preexisting path cannot continue upwards. If the invariant held, continuing demotion path along the same vertices would ensure a 3-child during rebalancing -- a contradiction with rank invariant holding. 
\end{myproof}

With the path rules established, we proceed with an overview of implementation of procedures.

To keep track of all modifying paths, three extra bits are added to each vertex to identify if it is a top of promotion, demotion or two demotion paths. A pointer to bottom of each modifying path is also added, being null for vertices where not enough paths are available.

When descending down the tree, it is easy to determine where we leave the modifying path. When we compare the value of a vertex to searched key, we also compare to the end of modifying path and if the comparison outcomes differ, we leave the path at this vertex. Rank of vertices that belong to a modifying path can be inferred rather easily. If we know that a promotion path continues here, the value should be increased by one. If two demotion paths continue through, rank should be reduced by 2 and rank of its child not on the demotion path should be decreased be 1. If one demotion path continues through, the type of demotion is determined from the rank of the child not on the path. The only tricky part is the second case of demote where rank of vertices not on the path is decreased, although this poses a mere inconvenience for implementation, not a fundamental problem.

Next we describe auxiliary procedures used for keeping the tree balanced. These are relied upon by insert and delete.

\subsection{Cut top from modifying path}

A vertex that is the top of a modifying path is excluded from that path and its rank is adjusted. Then, if there is one last vertex remaining on the path, it has its rank adjusted and the path is destroyed completely.

\subsection{Walk to key}

Walk descends from the root of the tree to a specified key and builds a temporary {\em walking path} of vertices corresponding to those visited on the path from root to searched key. Modifications by paths are written to those temporary vertices. The type of modification is stored and also pointer to the top of that path in the walking path.

\subsection{Remove from demotion path}

This procedure is given a vertex $V$ on a walking path that is also part of a demotion path. The demotion path is split in two parts with $V$ separating them. If $V$ is part of two demotion paths this is done for both of those. This is achieved by modifying the vertex on top the demotion path and the vertex directly below $V$. There can only be one part if $V$ lies on the end of the demotion path. Should any of the new parts be a single vertex, it is demoted immediately. $V$ is demoted.

The changes are written both to the walking path and the ranked tree. Since the top of modifying path changed for the bottom part, top must be updated for those vertices in the walking path.

We see that this operation will update at most three vertices in the tree.

\subsection{Remove from promotion path}

Very analogous to previous operation with the only difference being promotion in place of demotion.

We see that this operation will also update at most three vertices in the tree.

\subsection{Rotate promotion}

This procedure receives a vertex on a walking path where the rank rule is violated and (double) rotation is necessary. It is assumed that it is not part of any modifying path and that its $0$-child on the walking path is neither. 

It should be noted that the specified vertex will have a child on the walking path. This follows from the fact that a parent of a newly inserted leaf will not be a $(0,2)$ vertex.

The kind of rotation needed can be deduced locally. Should there be additional modifications from modifying paths, these will be apparent (as marked as such at the top of corresponding modifying path). 

Required rotation is performed, it may be needed to remove one more vertex from modifying path in case of double rotation.

It suffices to write all changes to the original ranked tree since balancing will end by this step and walking path will be abandoned. 
This procedure returns a new root of the tree, this is either the the original root or the vertex that was rotated above.

\subsection{Rotate demotion}

This procedure receives a vertex $V$ on a walking path where the rank rule is violated and (double) rotation is necessary. It is assumed that it is not part of any modifying path and that its $3$-child is neither. 

Again, if the walking path continues below $V$, the 3-child must lay on it. The kind of rotation is deduced from the other child of $V$ and its children. If those vertices need to be rotated and lie on a modifying path, they need to be removed from it. It can be seen that in such a case they form tops of their paths and the special operation described above can be used.

It suffices to write all changes to the original ranked tree since balancing will end by this step and walking path will be abandoned. New root is returned.

\subsection{Pass promotion up}

A walking path is passed as argument, the bottom vertex needs to be promoted. The promotion is propagated above the path while needed. One of the following actions is taken at each vertex $V$ based on its relation to modifying paths.

\begin{itemize}

\item {\bfseries $V$ is not part of any path} First we check whether promote or rotation will be needed. If rotation is needed, perform it. If promote is needed, the process proceeds to parent vertex.

\item {\bfseries $V$ has its rank demoted by parent} Parent must be removed from all paths it is on. If the rank is decreased by two, rotation will be required. Top of promotion path is marked.

\item {\bfseries $V$ is part of a promotion path} If $V$ is a (1,1)-vertex, mark its son as top of promotion path, remove $V$ from promotion path. Otherwise mark $V$'s son's son as top of promotion path, write $V$'s son's promotion directly, remove $V$ from promotion path and proceed with promote rotation. 

\item {\bfseries $V$ is part of one demotion path} If $V$ is a (1,1)-vertex, mark its son as top of demotion path, remove $V$ from demotion path. Otherwise mark $V$'s son's son as top of demotion path, write $V$'s son's demotion directly, remove $V$ from promotion path and proceed with promote rotation.

\item {\bfseries $V$ is part of two demotion paths} This vertex is removed from both demotion paths and its son becomes the top of new promotion path.

\end{itemize}

\subsection{Pass demotion up}

A walking path is passed as argument, the bottom vertex is not a leaf and needs to be demoted. The demotion is propagated above the path while needed. One of the following actions is taken at each vertex $V$ based on its relation to modifying paths.

\begin{itemize}

\item {\bfseries $V$ is not part of any path} If $V$ does not have a 3-son, mark the predecessor as top of demotion path. If the last vertex is reached, mark it as top of demotion path.  Otherwise inspect ranks of children and determine whether rotation or demotion is needed. In case of demotion, continue to next vertex. In case of rotation, write demote of $V$'s son, mark $V$'s son's son as top of demotion path, proceed with demote rotation.

\item {\bfseries $V$ has its rank demoted by parent} Remove $V$'s parent from all demotion paths. Mark either this vertex or its son as a top of new demotion path.

\item {\bfseries $V$ is part of a promotion path} If $V$ is a (2,2)-vertex, mark its son as top of demotion path, remove $V$ from promotion path. Otherwise mark $V$'s son's son as top of demotion path, write $V$'s son's demotion directly, remove $V$ from promotion path and proceed with demote rotation. 

\item {\bfseries $V$ is part of one demotion path} If $V$ is a (2,2)-vertex, mark its son as top of demotion path, remove $V$ from demotion path. If second demote is required, continue to parent. Otherwise mark $V$'s son's son as top of demotion path, write $V$'s son's demotion directly, remove $V$ from promotion path and proceed with demote rotation. 

\item {\bfseries $V$ is part of two demotion paths} A rotation is required, remove $V$ from both demotion paths. Otherwise mark $V$'s son's son as top of demotion path, write $V$'s son's demotion directly, remove $V$ from promotion path and proceed with demote rotation.

\end{itemize}

In fact, should the path be too short (1 vertex), instead of marking it, it is possible to directly change vertex ranks.

\subsection{Balance walking path}

We start ascension from the bottom of the walking back to the root. We use the walking path to be aware of true ranks. If promotion is needed, it passed up. If demotion is needed it is passed up. If rotation is needed, it is performed. New top of the tree is returned. 

The following operations form the interface of the WAVL tree.

\subsection{Find, Predecessor \& Successor}

Since promote and demote steps do not change the structure of the tree (they only change the ranks), all searching operations remain identical to their counterparts in standard algorithm.

\subsection{Insert}

A new vertex is inserted in the obvious place in the tree. A walking path is obtained to the parent and balanced subsequently.

\subsection{Delete}

The process is very similar to insert. First, if the vertex cannot be deleted directly, the vertex bearing the next lower/higher key must be identified and the keys \& values exchanged. Then search for the key about to be deleted (which is now in a vertex with only one child) starts from the root. If the vertex found is a leaf, walking path is obtained to the parent of it and balanced. Otherwise its child is deleted and walking path to it is balanced.

The proposition below follows from this algorithm.

\begin{prop}
WAVL tree can be stored in such a way that $\mathcal{O}(1)$ modifications of the structure are needed for insert and delete, while $\mathcal{O}(n)$ space is needed to store the structure and complexity of operation with the tree remains $\mathcal{O}(\log n)$.
\end{prop}

\begin{figure}
\begin{center}
\begin{tikzpicture}[sibling distance=8pt]
\Tree
[
.~
\edge node[auto=right] {1};
[ 
    .~
    \edge node[auto=right] {1};
    [  .~ \edge node[auto=right] {1}; [.~ \edge node[auto=right] {0}; X \edge[blank]; \node[blank]{}; ] \edge node[auto=left] {1}; ~ ]
    \edge node[auto=left] {1};
    [ .~ \edge node[auto=right] {1}; ~ \edge node[auto=left] {1}; ~ ]
] 
\edge node[auto=left] {1};
[ 
    .~
    \edge node[auto=right] {1};
    [  .~ \edge node[auto=right] {1}; ~ \edge node[auto=left] {1}; ~ ]
    \edge node[auto=left] {1};
    [ .~ \edge node[auto=right] {1}; ~ \edge node[auto=left] {1}; ~ ]
]
]
\end{tikzpicture}
\qquad
\begin{tikzpicture}[sibling distance=8pt]
\Tree
[
.\node[red]{~};
\edge[red] node[auto=right] {1};
[ 
    .\node[red]{~};
    \edge[red] node[auto=right] {1};
    [  .\node[red]{~}; \edge[red] node[auto=right] {1}; [.\node[red]{~}; \edge node[auto=right] {1}; X \edge[blank]; \node[blank]{}; ] \edge node[auto=left] {2}; ~ ]
    \edge node[auto=left] {2};
    [ .~ \edge node[auto=right] {1}; ~ \edge node[auto=left] {1}; ~ ]
] 
\edge node[auto=left] {2};
[ 
    .~
    \edge node[auto=right] {1};
    [  .~ \edge node[auto=right] {1}; ~ \edge node[auto=left] {1}; ~ ]
    \edge node[auto=left] {1};
    [ .~ \edge node[auto=right] {1}; ~ \edge node[auto=left] {1}; ~ ]
]
]
\end{tikzpicture}
\end{center}
\caption{Exemplar promotion path creation}
\end{figure}

\begin{figure}
\begin{center}
\begin{tikzpicture}[sibling distance=8pt]
\Tree
[ \edge node[auto=right]{2}; [.Y
    \edge node[auto=right] {3};
    \node[draw,triangle]{~};
    \edge node[auto=left] {2};
    [.~ \edge node[auto=right] {2}; \node[draw,triangle]{~}; \edge node[auto=left] {2}; \node[draw,triangle]{~};]
] ]
\end{tikzpicture}
\qquad
\begin{tikzpicture}[sibling distance=8pt]
\Tree
[ \edge node[auto=right]{2}; [.Y
    \edge node[auto=right] {2};
    \node[draw,triangle]{~};
    \edge node[auto=left] {1};
    [.~ \edge node[auto=right] {2}; \node[draw,triangle]{~}; \edge node[auto=left] {2}; \node[draw,triangle]{~};]
] ]
\end{tikzpicture}
\qquad
\begin{tikzpicture}[sibling distance=8pt]
\Tree
[ \edge node[auto=right]{2}; [.Y
    \edge node[auto=right] {3};
    \node[draw,triangle]{~};
    \edge node[auto=left] {1};
    [.~ \edge node[auto=right] {2}; \node[draw,triangle]{~}; \edge node[auto=left] {2}; \node[draw,triangle]{~};]
] ]
\end{tikzpicture}
\qquad
\begin{tikzpicture}[sibling distance=8pt]
\Tree
[ \edge node[auto=right]{2}; [.Y
    \edge node[auto=right] {2};
    \node[draw,triangle]{~};
    \edge node[auto=left] {1};
    [.~ \edge node[auto=right] {1}; \node[draw,triangle]{~}; \edge node[auto=left] {1}; \node[draw,triangle]{~};]
] ]
\end{tikzpicture}
\end{center}
\caption{The only case where double demotion can arise}
\end{figure}

\subsection{Keys in place of pointers for path ends}

Layout of vertex can be modified slightly. It is possible to store key of the vertex where modifying path ends instead of pointer to it.

The drawback is that whenever a key is changed, path to the root must be checked for vertices referring the old key as a path end. Fortunately, this seldom happens. Prior to performing most operations, vertices are excluded from any modifying paths. In fact, only delete procedure must be changed to explicitly check for references. (Due to swapping keys.)

The aforementioned complication causes this approach to be less comfortable, there is no incentive to use it without other reasons. The benefit stems from decreasing the number of pointers per vertex, which will be useful during conversion to persistent data structure.